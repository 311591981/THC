\documentclass[letterpaper, 12pt, oneside]{article}
\usepackage[utf8]{inputenc}

\title{Resumen\\Clase - 03}
\author{Emilio López Sotelo}
\date{9/01/2019}

\begin{document}
	\maketitle
	La tercera clase comenzo con la introduccion a nuevos comandos y a un argumento nuevo para el comando "ls" asi como tambien el uso de ">" en bash. Los nuevos comandos que se vieron fueron:
	\begin{itemize}
		\item cat, el cual tiene como funcion mostrar en la pantalla del interprete el contenido del archivo.
		\item mkdir, el cual crea un directorio nuevo en la ruta actual. Si se quiere crear un directorio y la ruta completa a ese nuevo directorio no existe se debe usar el argumento "-p" para crear los directorios faltantes, por ejemplo: si la ruta actual es home/Documents/ y la carpeta Documents/ esta vacia pero se quiere crear el directorio home/Documents/a/b/c/d/ se debe escribir el comando "mkdir -p /a/b/c/d/" para poder crear la carpeta d/
		\item rmdir, el cual elimina un directorio vacio
	\\
	\end{itemize}
	El nuevo argumento para el comando "ls" fue "-la" el cual sirve para enlistar los archivos y directorios ocultos dentro de la ruta en donde se esta. El caracter ">" tiene como proposito el guardar el resultado de un comando dentro de un archivo de texto plano.
	\\
	\\
	Despues de esto aprendimos una forma de evitar el tener que salir de bash para escribir un comentario al momento de ejecutar el comando "git commit", simplemente hay que ejecutar el comando con el argumento "-m" seguido del comentario, por ejemplo:
	\begin{itemize}
		\item git commit -m "cambios al archivo abc.txt"
	\\
	\end{itemize}
	Tambien tuvimos nuestro primer encuentro con los editores de texto integrados a las distribuciones de Linux que son "vi" y "gedit". Para comenzar a usarlos basta con escribirlos en bash como si fueran un comando. Hacer esto crea un archivo de texto plano .txt nuevo en la ruta actual, si se quiere editar un archivo ya existente se debe escribir el nombre del editor de texto seguido de la ruta del archivo desde la ruta actual. Por ejemplo, si uno se encuentra en home/ y quisiera editar el archivo abc.txt que se encuentra en la ruta home/Documents/abc.txt usando gedit uno tendria que escribir el comando "gedit /Documents/abc.txt". En cuanto a los editores "gedit" cuenta con una interfaz grafica similar a la de Wordpad en Windows, es decir que contiene botones para abrir opciones del archivo como podrian ser guardar, abrir, etc. Por otra parte "vi" es un interprete de comandos, esto quiere decir que para empezar a escribir o guardar un archivo y salir hay que presionar una combinacion de caracteres en el teclado. Algunos de estos comandos de vi son:
	\begin{itemize}
		\item i, para empezar a escribir
		\item Esc, para salir del modo de edicion
		\item w + Intro, para guardar el archivo, si es la primera vez que se guarda forzosamente hay que escribir el nombre del archivo seguido de haber presionado "w" y entonces presionar "Intro"
		\item q + Intro, para salir de vi, si uno no ha guardado su archivo "vi" no lo dejara salir y en este caso se debe presionar "q! + Intro" para salir sin guardar
	\\
	\end{itemize}
	Finalmente se termino la clase con una discusion sobre como abordar y resolver un problema ademas de una pequeña primera introduccion a Python. La discusion tuvo como objetivo el saber identificar un problema y los pasos a seguir para poder resolverlo de manera satisfactoria. La siguiente enumeracion funciona entonces como un esquema general para resolver un problema con la ayuda de la computacion:
	\begin{enumerate}
		\item Definir el problema, es decir que se quiere resolver
		\item Delimitar el problema, es decir analizar como se puede resolver y que herramientas se encuentran disponibles para resolverlo
		\item Buscar soluciones ya existentes para el mismo problema o en su caso problemas similares
		\item Realizar soluciones para casos especificos y describir detalladamente como se llego a esas soluciones
		\item Tratar de encontrar una solucion general que sea aplicable en todos los distintos casos del mismo problema
	\\
	\end{enumerate}
	Despues de este esquema se planteo un problema especifico a resolver, inspirado en los temas vistos en las clases de fisica de la preparatoria. El problema que se escogio fue el de encontrar la posicion de un objeto como funcion del tiempo dentro del escenario del tiro parabolico, este fenomeno esta dado por la ecuacion:
	\begin{itemize}
		\item $y(t) = V_{0} - 1/2gt^{2}$, donde " y(t) " es la posicion como funcion del tiempo, " $V_{0}$ " es la velocidad inicial del proyectil, " g " es la aceleracion a causa de la gravedad y " t " es el tiempo de vuelo del proyectil
	\\
	\end{itemize}
	Es importante anotar que aunque la funcion esta definida para todo el conjunto $\Re$ en un contexto de fisica solo nos interesan los valores donde t$\geq$0 ya que no nos interesan tiempo negativos, es decir lo que ocurre antes de que el proyectil entre en movimiento. Podemos entonces definir nuestro nuevo dominio como t$\in[0,2V_{0}/g]$. Tampoco nos interesan los valores donde y$<$0 ya que consideramos que en esas posiciones nosotros como observadores ya no tenemos visibilidad sobre el proyectil. Estas dos anotaciones son ejemplos de como se puede delimitar el problema.
\end{document}