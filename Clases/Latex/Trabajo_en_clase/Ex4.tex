\documentclass{beamer}
\usepackage{graphicx}
\usepackage[utf8]{inputenc}
\usepackage[spanish]{babel}
\graphicspath{{Imagenes/}}
%\usetheme{Antibes}
%\usetheme{AnnArbor}
%\usetheme{Berkeley}
%\usetheme{CambridgeUS}
%\usetheme{Goettingen}
\usetheme{Bergen}

%Esto únicamente se utiliza para el tema "Bergen"
\def\insertauthorindicator{Autor}
\def\insertdateindicator{Fecha}

\title{Taller de Herramientas Computacionales}
\author{Emilio López Sotelo}
\date\today

\begin{document}
	\maketitle
		\begin{frame}
		\frametitle{Mi Primera Presentación en LaTeX}
		\transdissolve
		\includegraphics[scale=0.15]{UNAM.png}
		\end{frame}
		\begin{frame}
		\frametitle{Segunda Diapositiva}
		Esta es mi segunda diapositiva
		\end{frame}
		\begin{frame}[fragile]
		\begin{verbatim}
		#!/usr/bin/python2.7
		# -*- coding: utf-8 -*-
		
		'''
		Emilio López Sotelo, 311591981
		Taller de Herramientas Computacionales
		--------------------------------------------------------------------------------
		Este es un programa que utilizamos para aprender como ejecutar un programa en
		python2.7 desde bash.
		'''
		
		print "Hoy es miércoles"
		
		#x = 10.5; y = 1.0/3; z = 15.3
		x,y,z=10.5,1.0/3,15.3
		
		H = """
		El punto en R3 es:
		(x,y,z) = (%.2f,%g,%G)
		\end{verbatim}
		\end{frame}
		\begin{frame}[fragile]
		\begin{verbatim}
		""" % (x,y,z)
		
		print H
		
		G ="""
		El punto en R3 es:
		(x,y,z) = ({laX:.2f},{laY:g},{laZ:G})
		
		""".format(laX=x,laY=y,laZ=z)
		
		print G
		
		import math
		from math import sqrt as s
		
		
		x=16
		
		x=input("Cual es el valor al que le quieres calcular la raíz?\n")
		\end{verbatim}
		\end{frame}
		\begin{frame}[fragile]
		\begin{verbatim}
		print "La raíz cuadrada de %.2f es:\n%f" % (x,s(x))
		
		y=input("Cual es el valor al que le quieres calcular la raíz?\n")
		
		print "La raíz cuadrada de %.2f es:\n%f" % (y,s(y))
		\end{verbatim}
		\end{frame}
\end{document}