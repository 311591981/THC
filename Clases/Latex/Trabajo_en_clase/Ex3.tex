\documentclass{book}
\usepackage[spanish]{babel}
\usepackage[utf8]{inputenc}
\usepackage{biblatex}
\usepackage{hyperref}

\title{\Huge Taller de Herramientas Computacionales}
\author{Emilio López Sotelo}
\date{17/01/2019}

\begin{document}
	\maketitle
	\tableofcontents
	\section*{Introducción}
	Este libro es para fortalecer el conocimiento de la materia Taller de Herramientas Computacionales.
	\\
	\url{www.google.com}
	\\
	\hyperref[Google]{www.google.com}
	
	\chapter{Uso básico de Linux}
	\section{Distribuciones de Linux}
	\section{Comandos}
	
	\chapter{Introducción a LaTeX}
	
	\chapter{Introducción a Python}
	\begin{verbatim}
	#!/usr/bin/python2.7
	# -*- coding: utf-8 -*-
	
	'''
	Emilio López Sotelo, 311591981
	Taller de Herramientas Computacionales
	--------------------------------------------------------------------------------
	Este es un programa que utilizamos para aprender como ejecutar un programa en
	python2.7 desde bash.
	'''
	
	print "Hoy es miércoles"
	
	#x = 10.5; y = 1.0/3; z = 15.3
	x,y,z=10.5,1.0/3,15.3
	
	H = """
	El punto en R3 es:
	(x,y,z) = (%.2f,%g,%G)
	""" % (x,y,z)
	
	print H
	
	G ="""
	El punto en R3 es:
	(x,y,z) = ({laX:.2f},{laY:g},{laZ:G})
	""".format(laX=x,laY=y,laZ=z)
	
	print G
	
	import math
	from math import sqrt as s
	
	
	x=16
	
	x=input("Cual es el valor al que le quieres calcular la raíz?\n")
	
	print "La raíz cuadrada de %.2f es:\n%f" % (x,s(x))
	
	y=input("Cual es el valor al que le quieres calcular la raíz?\n")
	
	print "La raíz cuadrada de %.2f es:\n%f" % (y,s(y))
	\end{verbatim}
	
	\input{Prueba01.py}
	
	\section*{Orientación a Objetos}
	
	\begin{thebibliography}{9}
		%\bibitem{Computación}
		Autor bla bla bla
		\\
		%\textit{Cualquier cosa}
		bla bla bla
	\end{thebibliography}
\end{document}