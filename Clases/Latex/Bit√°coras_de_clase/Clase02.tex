\documentclass[letterpaper, 12pt, oneside]{article}
\usepackage[utf8]{inputenc}

\title{\Huge \textbf{Resumen de clase\\02}}
\author{\textbf{Emilio López Sotelo}}
\date{\textbf{8/01/2019}}

\begin{document}
	\maketitle
	La segunda clase del curso la tomamos por primera vez en un taller de cómputo, esta fue la primera interacción en clase que tuvimos con un equipo físico donde pudimos, entre otras cosas, empezar a probar los comandos que habíamos visto la clase anterior. En mi caso yo llevé mi propia laptop personal por lo que un primer paso fue instalar una maquina virtual en mi computadora para poder virtualizar una distribución de Linux sin tener que instalar el sistema operativo a la par de Windows10. La maquina virtual que decidí ocupar fue Oracle VM Virtual Box y la distribución instalada en ella fue la versión mas reciente de Ubuntu64bits. Se nos explico que un core es equivalente a un CPU o procesador y nos fue recomendado solo instalar maquinas virtuales en equipos con 4 o mas de ellos, en mi caso mi computadora cuenta con 8 cores por lo que no tuve problema alguno.
	\\
	\\
	Ya con el equipo listo empezamos a probar algunos de los comandos vistos la clase pasada, algunos de ellos fueron:
	\begin{itemize}
		\item \textit{top}
		\item \textit{cd}, se nos explico que el argumento ".." sirve para subir al directorio anterior al actual y además pudimos por primera vez ver algunos directorios en Linux, algunos de ellos son:
		\begin{itemize} 
			\item \textit{lib 64}: donde están las bibliotecas de 64bits.
			\item \textit{lib}: donde están las bibliotecas de 32bits.
			\item \textit{home}: el directorio del usuario.
			\item \textit{media}: el directorio que contiene la información de medios externos como una memoria USB o un CD.
		\end{itemize}
		\item \textit{ls}, aprendimos que el argumento "-l" hace que la lista proporcionada contenga información sobre los archivos y directorios.
		\item \textit{df -lh}
		\item \textit{uname -a}
		\item \textit{touch}
		\item \textit{set}, vimos un ejemplo concreto de una variable de entorno que en este caso fue "LANG=esMX.UTF-8" la cual refiere al idioma y tipo de caracteres que están en configuración del equipo. Ademas se nos mostró como "|" sirve para redirigir el resultado de un comando a otro comando, en este caso se utilizo la combinacion "set $|$ less".
		\item \textit{chmod}, cambiamos los permisos de un archivo recién creado de 755 a 700.
		\item \textit{file}
		\
	\end{itemize}
	Para terminar la clase se nos explico lo que es Github, el cual es un servidor de Git que a su vez es un servicio en la nube que te permite subir, bajar e intercambiar información. Posteriormente ingresamos a la pagina web de Github para crear nuestro propio repositorio. La clase concluyo con los pasos para instalar Github en nuestra distribución de Linux y como clonar nuestro repositorio a la computadora así como también el como agregar nuevos archivos y directorios a nuestro repositorio desde la computadora. Los pasos a seguir fueron ejecutar los comandos a continuación en este mismo orden:
	\begin{enumerate}
		\item sudo apt install git, sirve para instalar git.
		\item sudo apt upgrade
		\item sudo apt update
		\item git init, sirve para iniciar el cliente de git.
		\item git config --global user.email "els1993@ciencias.unam.mx", sirve para especificar que correo asociado a git se va a utilizar.
		\item git config --global user.name "311591981", sirve para especificar que usuario asociado a git se va a utilizar.
		\item cd Documents, "Documents" es la carpeta donde quise clonar mi repositorio
		\item git clone URL, sirve para copiar mi repositorio al directorio Documents y URL es la URL de mi repositorio.
		\item cd THC, sirve para entrar al directorio "THC" que es mi repositorio local.
		\item git add *, el "*" sirve para agregar todos los archivos y directorios, de manera alternativa se puede especificar la ruta de lo que se quiere agregar
		\item git status, este comando sirve para ver el estado del repositorio, es decir que cambios estan en espera para realizarse. Cabe aclarar que este comando puede utilizarse en cualquier momento, por ejemplo tambien despues de utilizar el comando "git commit"
		\item git commit, aquí se escribe un comentario referente a los cambios que se van a hacer
		\item git push, sube los cambios agregados a la nube
		\item ingresas tu usuario, sirve para autentificar 
		\item ingresas tu contraseña, sirve para autentificar	
	\end{enumerate}
\end{document}